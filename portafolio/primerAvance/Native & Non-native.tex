\documentclass{article}
\usepackage{hyperref}

\title{Native and Non-Native Apps}
\author{}
\date{}

\begin{document}
	
	\maketitle
	
	What’s the difference between native apps and non-native apps?
	
	Breaking the mobile apps into 2 top-level categories from an overview, we got native and non-native apps. And then into the non-native, there are numerous sub-categories (hybrid, cross-platform and progressive web apps).
	
	A native app is built entirely with technologies that are specifically designed to leverage the mobile OS and hardware functions from the device where it's made for. In fact, the libraries used in native apps are designed to directly access all of the classes, objects, functions, and methods of the source code.
	
	For Native Android, apps are written in either Java or Kotlin.
	For iOS, apps are written in either Swift or Objective C.
	
	Non-native mobile apps are written in higher-level programming languages/frameworks and do not have direct access to all of the OS functions and hardware components.
	
	The main idea is that a native app is a mobile software that is made with the tools originally intended to work with its environment, while a non-native app is a mobile software developed using technologies or tools not specifically tied to the native environment of the device. In native apps, the code runs directly on the application. It is converted into binary code and executes from there. For non-natives, the source code is saved inside the app itself; when the app initializes, it reads the code from the source code saved file.
	
	\section*{Bibliography}
	\begin{itemize}
		\item Bavosa A. (2020). \textit{Native vs Non-Native Mobile Apps — What’s the Difference?} Extracted from: \url{https://medium.com/swlh/native-vs-non-native-mobile-apps-whats-the-difference-b3a641e06f52}
		\item Nader D. (2019). \textit{React Native in Action.} Manning Editorial.
	\end{itemize}
	
\end{document}
