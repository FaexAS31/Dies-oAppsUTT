\documentclass{article}
\usepackage[utf8]{inputenc}

\title{The Basics of Mobile App Development}
\author{Fabian Mendoza Contreras}
\date{09-01-2024}

\begin{document}
	
	\maketitle
	
	As soon as the sun takes the lead, most people do not blink twice before their eyes reach their mobile phones. A significant number of mobile apps have assumed a central role in our daily lives. As a result, the mobile app industry is growing exponentially, leading to a crowded competitive market that also necessitates a better understanding of this area and its best practices to stay in tune.
	
	To embark on the path to becoming a programmer with expertise, it is essential to acquire fundamental skills:
	
	\begin{enumerate}
		\item \textbf{Language Mastery:} Developers commonly use Java and XML to write code for Android apps due to their correlation with objects and classes, collections, inheritance and interfaces, packages, and concurrency.
		
		\item \textbf{Selection of the Right App Development Tools and Environment:} It is important to know about Integrated Development Environments (IDEs) and how a developer can use their full potential for their objectives.
		
		\item \textbf{Working with App Components:} Components are often seen as the building blocks of an Android App. Beginners should know how to work with these five basic components:
		\begin{itemize}
			\item Activities: A component that represents a UI (single-screen).
			\item Services: A component that performs background work for remote processes.
			\item Content providers: Components that manage shared data between files, the web, and the database.
			\item Broadcast receivers: A component that responds to pan-system announcements.
			\item Activating components: Also known as "intent," it binds individual components at runtime.
		\end{itemize}
		
		\item \textbf{Android Applications, Threads, and Tasks:} In the market, there are countless devices and operating systems from an exaggerated number of brands, creating a constant need in Android development for frequent tweaks and upgrades to allow a smooth user experience across all devices, regardless of the OS version it has been downloaded. Developers must ensure that threads are not blocked to maintain smooth operations in computational tasks, I/O, and network activities. This expertise helps developers create high-performing applications that excel in functionality and user satisfaction.
	\end{enumerate}
	
	\section*{Bibliography}
	\begin{itemize}
		\item Vasconcelos, H. (2016). \textit{Asynchronous Android Programming. 2nd Edition.} Packt Publishing.
		\item Philips, B. (2019). \textit{Android Programming: The Big Nerd Ranch Guide. 4th Edition.} Big Nerd Ranch Guides.
	\end{itemize}
	
\end{document}
